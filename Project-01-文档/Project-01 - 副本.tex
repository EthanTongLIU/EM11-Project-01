\documentclass[UTF8, 12pt, a4paper]{ctexart}

\usepackage{CJK}
\usepackage{amsmath, amssymb, amsfonts, array} % <-- 数学宏包, 数学样式
\usepackage{times}\usepackage{mathptmx}
\usepackage{latexsym} % <-- 符号
\usepackage{graphicx} % <-- 颜色和图形, 增强图形支持,文本旋转和缩放
\usepackage{subfigure} % <-- 插入并列子图
\usepackage[dvipsnames,usenames]{color} % <-- 颜色
\usepackage{xcolor} % <-- 颜色
\usepackage{textcomp} % <-- 特殊字符
\usepackage{float} % <-- 浮动
\usepackage{longtable} % <-- 长表格
\usepackage{multicol, multirow} % <-- 多列, 多行, 分栏
\usepackage{lipsum} % <-- 随机文本
%\usepackage{fontspec} % <-- 字体设置, 需要xelatex编译
\usepackage{setspace} % <-- 使用间距宏包

% ===============================
%  ----- 以下设置定理环境 -----
% ===============================
\newtheorem*{lemma}{引理} 
\newtheorem*{definition}{定义} 
\newtheorem*{theo}{定理} 

% ===============================
%  ----- 以下设置链接样式 -----
% ===============================
\usepackage[colorlinks, linkcolor=red, anchorcolor=blue, citecolor=green]{hyperref}

% ===============================
%  ----- 以下设置图片路径 -----
% ===============================
\graphicspath{{Figures/}}

% ===============================
%  ----- 以下设置页面边距 -----
% ===============================
\usepackage[top=1in, bottom=1in, left=1in, right=1in]{geometry}

% ===============================
%  ----- 以下设置页眉页脚 -----
% ===============================
\usepackage{fancyhdr}
\usepackage{lastpage} % <-- 获取总页数
\pagestyle{fancy}
\fancyhead[R]{\CJKfamily{kai} \bfseries EM11~Project-01~数值积分} % <-- 右页眉
\fancyhead[L]{\CJKfamily{hei} \bfseries \leftmark} % <-- 左页眉, 显示对应章标题
%\fancyfoot[C]{Page \thepage\ of \pageref{LastPage}} % <-- 中页脚, 当前页  of 总页数
%\fancyfoot[R]{\CJKfamily{song} 中欧航空工程师学院} % <-- 右页脚
%\fancyfoot[L]{\CJKfamily{kai} 中国民航大学} % <-- 左页脚

\renewcommand{\headrulewidth}{0.4pt} % <-- 上方装饰横线
%\renewcommand{\footrulewidth}{0.4pt} % <-- 下方装饰横线

\begin{document}
\begin{spacing}{1.5}%%行间距变为double-space

% --- >> 以下是文章标头信息 << ---
\title{ \CJKfamily{kai} Project-01~数值积分 } % <-- 文章标题
\author{刘通(Tong~LIU)} % <-- 作者信息
\date{2017~年~9~月~7~日} % <-- 删除自动创建的日期, 如果需要添加日期, 请自行填写
\maketitle % <-- 生成文章标头
\tableofcontents % <-- 生成目录
\newpage
% --- >> 以下是文章主体内容 << ---
\part{算法原理}
\section{用\emph{拉格朗日插值多项式}来逼近被积函数}
已知对于 $ f $ 和给定的节点 $ \{ (x_i,y_i) \}_{i \in [0,n] } $ ,我们通过用这些节点进行插值
,以构造一个函数来逼近被积函数 $ f $,可以通过构造一个 $ n $ 次多项式函数来逼近 $ f $。

希望构造一个 $ n $ 次多项式函数:
$$ P_{n}(x) = a_0 + a_1 x \cdots + a_n x^n  $$
让其满足插值条件:$ P_{n}(x_i) = y_i~,~i=0,1,\cdots,n~$ ,这个多项式函数 $ P_{n} $ 就是
$ f $ 的 $ n $ 次插值多项式。

对于这给定的 $ n+1 $ 个节点来说,构成线性方程组:
\begin{align*}
  \left\{
  \begin{array}{rcl}
a_0 + a_1 x_0 + & \cdots & + a_n x_0^n = y_0 \\
a_0 + a_1 x_1 + & \cdots & + a_n x_1^n = y_1 \\
                & \cdots & \\
a_0 + a_1 x_n + & \cdots & + a_n x_n^n = y_n
  \end{array} \right.
\end{align*}

求解这个线性方程组很不方便,尤其当 $n$ 特别大的时候,但是由于其系数矩阵的行列式是范德蒙行列式,
由范德蒙行列式的性质可知,其行列式值不为零,故其系数矩阵可逆且解是唯一的。说明该插值多项式是唯一的。

所以可以通过构造插值基函数 $ l_i $ 来构造满足 $ f(x_i) = y_i~,~i=0,1,\cdots,n $ 的多项式函数。

令
\begin{equation}
\boxed{ f(x) \approx L_{n}(x) = \sum \limits _{i=0}^{n} y_i l_i(x) }
\label{eq:1}
\end{equation}

其中,插值基函数 $l_i$ 满足:
\begin{equation*}
l_i(x_k) = \left\{
\begin{array}{rcl}
0,& i = k\\
1,& i \neq k
\end{array} \right.
\end{equation*}
\begin{equation}
l_{i}(x) = \dfrac{ (x-x_0)\cdots(x-x_{i-1})(x-x_{i+1})\cdots(x_i-x_n) }{ (x_i-x_0)\cdots(x_i-x_{i-1})(x_i-x_{i+1})\cdots(x_i-x_n) } = \prod _{j=0,j \neq i}^{n} \dfrac{x-x_i}{x_i-x_j}
\end{equation}

式\eqref{eq:1}即为$n$次插值多项式。

\section{构造插值求积公式}

对式\eqref{eq:1}两边积分,以构造插值求积公式:
\begin{equation}
\boxed{
\int _a^b f(x)dx \approx \int _a^b L_n(x)dx = \int_a^b \sum \limits _{i=0}^n y_il_i(x)dx = \sum_{i=0}^n\left[ \int_a^b l_i (x) dx \right]y_i = \sum \limits _{i=0}^{n}A_{k}y_i
}
\end{equation}
其中,
\begin{equation}
A_k = \int_a^b l_i (x) dx = \int_a^b \dfrac{ (x-x_0)\cdots(x-x_{i-1})(x-x_{i+1})\cdots(x_i-x_n) }{ (x_i-x_0)\cdots(x_i-x_{i-1})(x_i-x_{i+1})\cdots(x_i-x_n) }dx
\end{equation}
就是插值求积公式的系数。
\section{从~Newton-Cotes~积分到~Gauss~积分}
由前面的介绍可以看出,插值积分公式最后的求积结果只和积分区间上节点的选取有关,那么如何选
取这些节点可以达到最好的效果呢?

\part{算法构建}

\end{spacing}
\end{document}
